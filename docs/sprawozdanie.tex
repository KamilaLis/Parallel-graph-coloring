% !TeX encoding = UTF-8
% !TeX spellcheck = pl_PL
\documentclass{article}
\newcommand\tab[1][1cm]{\hspace*{#1}}
\usepackage[]{polski}
\usepackage[utf8]{inputenc}
\usepackage{graphicx}
\usepackage{float}
\usepackage{amsmath}
\usepackage{geometry}
 
\usepackage{listings}
\usepackage{color}
\usepackage{hyperref}

\usepackage{multirow}
\usepackage{pdfpages}

\definecolor{codegreen}{rgb}{0,0.6,0}
\definecolor{codegray}{rgb}{0.5,0.5,0.5}
\definecolor{codepurple}{rgb}{0.58,0,0.82}
\definecolor{backcolour}{rgb}{0.95,0.95,0.92}

\lstdefinestyle{mystyle}{
	backgroundcolor=\color{backcolour},   
	commentstyle=\color{black},
	keywordstyle=\color{blue},
	numberstyle=\tiny\color{codegray},
	stringstyle=\color{codepurple},
	basicstyle=\footnotesize,
	breakatwhitespace=false,         
	breaklines=true,                 
	captionpos=b,                    
	keepspaces=true,                 
	%numbers=left,                    
	%numbersep=5pt,                  
	showspaces=false,                
	showstringspaces=false,
	showtabs=false,                  
	tabsize=2
}

\lstset{style=mystyle}
\date{}

\author{Katarzyna Dziewulska, Kamila Lis}

\title{Równoległe przetwarzanie grafu\\
	{\large Sprawozdanie 1. projektu}}
\bibliographystyle{plunsrt}

\begin{document}
	\maketitle
	W ramach projektu zostanie zaimplementowany algorytm kolorowania grafu w wersji zwykłej i równoległej. Sugerując się artykułem \cite{article} postanowiono wykorzystać algorytm LDF (Largest-Degree-First). Wersja równoległa zostanie zrealizowana w środowisku CUDA przy wykorzystaniu karty graficznej Nvidia.\\
	
	\bibliography{bibliografia}

 
%	\begin{table}[H]
%		\caption{Wartość średnia i mediana dla kolejnych cech w zbiorze uczącym}
%		\label{}
%		\begin{tabular}{r|c c c c c c c}
%			 & 2& 3 &4& 5& 6& 7& 8\\
%			 \hline
%			średnia  & 1.8679e-01 &  1.4839e-02 &  2.1045e-01 &  2.0882e-01 &  7.9658e+01  & 1.0604e+00  & 9.0846e-03\\
%			%\hline
%			mediana &  1.8259e-01 &  1.4785e-04 &  1.7434e-04 &  1.9996e-06 & -8.9358e-11  & 1.3626e-10 & -1.8427e-14 \\
%		\end{tabular} 
%	\end{table}


%	\begin{figure}[h]
%		\centering
%		\includegraphics[width=1\textwidth]{images/wybor_cech.png}
%		\caption{Wykres wartości cech 2 i 4.}
%		\label{24}	
%	\end{figure}

\end{document}