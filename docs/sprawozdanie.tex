% !TeX encoding = UTF-8
% !TeX spellcheck = pl_PL
\documentclass{article}
\newcommand\tab[1][1cm]{\hspace*{#1}}
\usepackage[]{polski}
\usepackage[utf8]{inputenc}
\usepackage{graphicx}
\usepackage{float}
\usepackage{amsmath}
\usepackage{geometry}
 
\usepackage{listings}
\usepackage{color}
\usepackage{hyperref}

\usepackage{multirow}
\usepackage{pdfpages}

\definecolor{codegreen}{rgb}{0,0.6,0}
\definecolor{codegray}{rgb}{0.5,0.5,0.5}
\definecolor{codepurple}{rgb}{0.58,0,0.82}
\definecolor{backcolour}{rgb}{0.95,0.95,0.92}

\lstdefinestyle{mystyle}{
	backgroundcolor=\color{backcolour},   
	commentstyle=\color{black},
	keywordstyle=\color{blue},
	numberstyle=\tiny\color{codegray},
	stringstyle=\color{codepurple},
	basicstyle=\footnotesize,
	breakatwhitespace=false,         
	breaklines=true,                 
	captionpos=b,                    
	keepspaces=true,                 
	%numbers=left,                    
	%numbersep=5pt,                  
	showspaces=false,                
	showstringspaces=false,
	showtabs=false,                  
	tabsize=2
}

\lstset{style=mystyle}
\date{}

\author{Katarzyna Dziewulska, Kamila Lis}

\title{Równoległe przetwarzanie grafu}
	%\\{\large Sprawozdanie 1. projektu}}
\bibliographystyle{plunsrt}

\begin{document}
	\maketitle
	W ramach projektu zostanie zaimplementowany algorytm kolorowania grafu w wersji zwykłej i równoległej. Sugerując się artykułem \cite{article} postanowiono wykorzystać algorytm LDF (Largest-Degree-First). Wersja równoległa zostanie zrealizowana w środowisku CUDA przy wykorzystaniu karty graficznej Nvidia.\\
	
	\section{Algorytm kolorowania}
	Kolorowanie wierzchołków polega na  przydzielaniu kolorów wierzchołkom tak, że by dwa sąsiednie wierzchołki otrzymały różne kolory. Najmniejsza liczba kolorów $k$ nazywana jest liczbą chromatyczną i oznaczana jako $X(G)$. W celu reprezentacji pokolorowania przyjęło się oznaczać kolory kolejnymi liczbami naturalnymi. Kolorowanie grafów najczęściej wykorzystywane jest do poszukiwania rozwiązań, w których unika się konfliktów. Strategie kolorowania grafu zależą od stawianych wymagań. Celem optymalnego kolorowania jest minimalizacja liczby użytych kolorów, podczas gdy kolorowanie zbalansowane dąży do zapewnienia podobnej liczby wierzchołków w każdym kolorze.
	\subsection{Largest First}
	Algorytm LDF (ang. \textit{Largest-Degree-First}) jest algorytmem sekwencyjnym, w którym wierzchołki są przeglądane i kolorowane według nierosnących stopni wierzchołkowych (wierzchołek o największym stopniu jest kolorowany pierwszy). Algorytm ma celu minimalizowanie maksymalnej liczby wykorzystanych kolorów.
	
	\subsection{Metoda równoległa}
	Równoległe metody kolorowania grafów są oparte na obserwacji, że każdy niezależny zbiór wierzchołków, czyli taki w którym nie ma żadnych wierzchołków sąsiadujących ze sobą, może być kolorowany równolegle. Różnice pomiędzy metodami sprowadzają się do sposobu wyboru niezależnego zbioru i samego kolorowania wierzchołków. Niezależny zbiór wierzchołków jest konstruowany jako podgraf indukowany zawierający jedynie niepokolorowane wierzchołki.
	
%For ecient implementation on distributed
%memory parallel computers, the information must also be local in the processor grid , so
%that the amount of communication is minimized. This means that the distribution of the
%graph over processors must be such that the number of edges crossing processor boundaries
%is minimized. This is the standard graph partitioning problem[]

	Algorytm LF może być zrównoleglony poprzez 	 znajdowanie zbióru niezależnych wierzchołków równolegle używając metody Luby (?) wybierania wierzchołków, których stopnie są lokalnymi maksymami.
	%[podobnie do Jonesa-Plassmanna tylko zamiast używania losowych wag do stworzenia niezależnego zbioru wagi są stopniami wierzchołków indukowanego podgrafu]. Liczby losowe są tylko używane do rozwiązania konfliktów między sąsiadującymi wierzchołkami o takich samych stopniach.
	Każdy z wierzchołków ma przydzielone trzy parametry:
	\begin{itemize}
		\itemsep0em
		\item swój stopień $deg(v)$,
		\item losową wartość $rndvalue(v)$, 
		\item listę zakazanych kolorów użytych już przez sąsiadów $usedcolor(v)$. 
	\end{itemize}   
	\begin{enumerate}
		\item losowo przydziel $rndvalue(v)$ oraz określ $deg(v)$ każdego wierzchołka,
		\item dla każdego wierzchołka sprawdź stopnie jego sąsiadów, jeśli aktualny posiada największy przydziel kolor,
		\item konflikty stopni rozwiązywane są przez wybór największej liczby $rndvalue(v)$
	\end{enumerate}
	\begin{lstlisting}
	
	U:=V
	while (|U|>0) do in parallel
		choose an independent set I from U
		color all vertices in I
		U:=U-I
	end do
	\end{lstlisting}
 
	\section{Struktury danych}
	
	
	\section{Projekty testów}
	\section{Założenia programu}
	
	\bibliography{bibliografia}

 
%	\begin{table}[H]
%		\caption{Wartość średnia i mediana dla kolejnych cech w zbiorze uczącym}
%		\label{}
%		\begin{tabular}{r|c c c c c c c}
%			 & 2& 3 &4& 5& 6& 7& 8\\
%			 \hline
%			średnia  & 1.8679e-01 &  1.4839e-02 &  2.1045e-01 &  2.0882e-01 &  7.9658e+01  & 1.0604e+00  & 9.0846e-03\\
%			%\hline
%			mediana &  1.8259e-01 &  1.4785e-04 &  1.7434e-04 &  1.9996e-06 & -8.9358e-11  & 1.3626e-10 & -1.8427e-14 \\
%		\end{tabular} 
%	\end{table}


%	\begin{figure}[h]
%		\centering
%		\includegraphics[width=1\textwidth]{images/wybor_cech.png}
%		\caption{Wykres wartości cech 2 i 4.}
%		\label{24}	
%	\end{figure}

\end{document}